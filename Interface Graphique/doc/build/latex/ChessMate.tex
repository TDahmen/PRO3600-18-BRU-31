%% Generated by Sphinx.
\def\sphinxdocclass{report}
\documentclass[letterpaper,10pt,french]{sphinxmanual}
\ifdefined\pdfpxdimen
   \let\sphinxpxdimen\pdfpxdimen\else\newdimen\sphinxpxdimen
\fi \sphinxpxdimen=.75bp\relax

\usepackage[utf8]{inputenc}
\ifdefined\DeclareUnicodeCharacter
 \ifdefined\DeclareUnicodeCharacterAsOptional
  \DeclareUnicodeCharacter{"00A0}{\nobreakspace}
  \DeclareUnicodeCharacter{"2500}{\sphinxunichar{2500}}
  \DeclareUnicodeCharacter{"2502}{\sphinxunichar{2502}}
  \DeclareUnicodeCharacter{"2514}{\sphinxunichar{2514}}
  \DeclareUnicodeCharacter{"251C}{\sphinxunichar{251C}}
  \DeclareUnicodeCharacter{"2572}{\textbackslash}
 \else
  \DeclareUnicodeCharacter{00A0}{\nobreakspace}
  \DeclareUnicodeCharacter{2500}{\sphinxunichar{2500}}
  \DeclareUnicodeCharacter{2502}{\sphinxunichar{2502}}
  \DeclareUnicodeCharacter{2514}{\sphinxunichar{2514}}
  \DeclareUnicodeCharacter{251C}{\sphinxunichar{251C}}
  \DeclareUnicodeCharacter{2572}{\textbackslash}
 \fi
\fi
\usepackage{cmap}
\usepackage[T1]{fontenc}
\usepackage{amsmath,amssymb,amstext}
\usepackage{babel}
\usepackage{times}
\usepackage[Sonny]{fncychap}
\usepackage[dontkeepoldnames]{sphinx}

\usepackage{geometry}

% Include hyperref last.
\usepackage{hyperref}
% Fix anchor placement for figures with captions.
\usepackage{hypcap}% it must be loaded after hyperref.
% Set up styles of URL: it should be placed after hyperref.
\urlstyle{same}

\addto\captionsfrench{\renewcommand{\figurename}{Fig.}}
\addto\captionsfrench{\renewcommand{\tablename}{Tableau}}
\addto\captionsfrench{\renewcommand{\literalblockname}{Code source}}

\addto\captionsfrench{\renewcommand{\literalblockcontinuedname}{continued from previous page}}
\addto\captionsfrench{\renewcommand{\literalblockcontinuesname}{continues on next page}}

\addto\extrasfrench{\def\pageautorefname{page}}

\setcounter{tocdepth}{1}



\title{ChessMate Documentation}
\date{mars 13, 2018}
\release{1}
\author{Thomas Dahmen, Oscar Bouvier, Jean Forissier, Raphael Macquet}
\newcommand{\sphinxlogo}{\vbox{}}
\renewcommand{\releasename}{Version}
\makeindex

\begin{document}

\maketitle
\sphinxtableofcontents
\phantomsection\label{\detokenize{index::doc}}


Contents:


\chapter{Page d’autodocumentation}
\label{\detokenize{autodoc:bienvenue-sur-cette-documentation}}\label{\detokenize{autodoc:page-d-autodocumentation}}\label{\detokenize{autodoc::doc}}

\section{echecs.py}
\label{\detokenize{autodoc:echecs-py}}
Voici l’autodocumentation d’une simple fonction de module
\index{move() (dans le module echecs)}

\begin{fulllineitems}
\phantomsection\label{\detokenize{autodoc:echecs.move}}\pysiglinewithargsret{\sphinxcode{echecs.}\sphinxbfcode{move}}{\emph{a}, \emph{b}, \emph{c}, \emph{d}}{}
Move a piece located on (a,b) to (c,d) if the movement is allowed by changing the values of plateau
Update dico\_position\_W, dico\_position\_B, position\_W, position\_B,wonW,wonB to make algorithm be coherent
Inverse the boolean value of tour\_blanc to simulate an alternative gameplay
\begin{quote}\begin{description}
\item[{Paramètres}] \leavevmode\begin{itemize}
\item {} 
\sphinxstyleliteralstrong{a} (\sphinxstyleliteralemphasis{int}) \textendash{} X axis of the piece we want to move

\item {} 
\sphinxstyleliteralstrong{b} (\sphinxstyleliteralemphasis{int}) \textendash{} Y axis of the piece we want to move

\item {} 
\sphinxstyleliteralstrong{c} (\sphinxstyleliteralemphasis{int}) \textendash{} X axis of the position we want to move the piece on

\item {} 
\sphinxstyleliteralstrong{d} (\sphinxstyleliteralemphasis{int}) \textendash{} Y axis of the position we want to move the piece on

\end{itemize}

\item[{Retourne}] \leavevmode
None

\item[{Type retourné}] \leavevmode
None

\end{description}\end{quote}

\end{fulllineitems}



\chapter{Complement}
\label{\detokenize{index:complement}}
Comme promis voici un petit exemple de documentation manuelle
\index{ma\_fonction() (fonction de base)}

\begin{fulllineitems}
\phantomsection\label{\detokenize{index:ma_fonction}}\pysiglinewithargsret{\sphinxbfcode{ma\_fonction}}{\emph{arg00}, \emph{arg01}, \emph{arg03='4'}}{}~\begin{quote}\begin{description}
\item[{Paramètres}] \leavevmode\begin{itemize}
\item {} 
\sphinxstyleliteralstrong{arg00} (\sphinxstyleliteralemphasis{int}) \textendash{} premier argument

\item {} 
\sphinxstyleliteralstrong{arg01} (\sphinxstyleliteralemphasis{dict}) \textendash{} second argument

\item {} 
\sphinxstyleliteralstrong{arg02} (\sphinxstyleliteralemphasis{str}) \textendash{} troisieme argument

\end{itemize}

\item[{Retourne}] \leavevmode
l’ensemble des elements demandes

\item[{Type retourné}] \leavevmode
list

\end{description}\end{quote}

\end{fulllineitems}




\renewcommand{\indexname}{Index}
\printindex
\end{document}